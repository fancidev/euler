\documentclass[10pt]{article}
\usepackage{amsmath}
\usepackage[utf8]{inputenc}
\setlength{\parskip}{1ex plus 0.5ex minus 0.2ex}

%\newcommand{\problem}[2][No]{\noindent{\large \sc Problem #1. #2.}}
\newcommand{\problem}[2][No]{\noindent{\sc Problem #1. #2.}}
\newcommand{\solution}{\noindent{\sc Solution}}
\newcommand{\method}[1]{\noindent{\sc Method #1}}
\newcommand{\complexity}{\noindent{\sc Complexity}}
\newcommand{\answer}{\noindent{\textsc{Answer}}}
\newcommand{\reference}{\noindent{\sc Reference}}
\newcommand{\seealso}{\noindent{\sc See Also}}

\newcommand{\BigO}{\mathcal{O}}


\begin{document}

\problem[133]{Prime Non-Factors of Repunits of $10^k$ Digits}

A number consisting entirely of ones is called a \emph{repunit}. We shall define $R(n)$ to be a repunit of length $n$. For example, $R(6) = 111111$.

Let us consider repunits of the form $R(10^k)$. Although $R(10)$, $R(100)$, or $R(1000)$ are not divisible by 17, $R(10000)$ is divisible by 17. Yet there is no value of $k$ for which $R(10^k)$ is divisible by 19. In fact, it is remarkable that 11, 17, 41, and 73 are the only four primes below 100 that can be a factor of $R(10^k)$.

Find the sum of all the primes below 100 000 that will never be a factor of $R(10^k)$.

\solution

Using results from Solution 132, we know that a prime number $p$ is a divisor of $R(n)$ if and only if $p=3$ and 3 divides $n$, or $p \ge 7$ and
\begin{equation}
10^{\gcd(n, p-1)} \equiv 1 \pmod{p} . \label{eq:133.1}
\end{equation}

Now consider $n$ of the form $10^k$. Obviously 3 does not divide $10^k$, so we are left with the condition $p \ge 7$ and
\begin{equation}
10^{\gcd(10^k, p-1)} \equiv 1 \pmod{p} . \label{eq:133.2}
\end{equation}
Let $(p-1) = 2^r \cdot 5^s \cdot d$ where $d$ does not contain any factor of 2 or 5. Then for sufficiently large $k$, $\gcd(10^k, p-1) = 2^r \cdot 5^s$. So a solution in $k$ to equation \eqref{eq:133.2} exists if and only if
\begin{equation}
10^{2^r \cdot 5^s} \equiv 1 \pmod{p} . \label{eq:133.3}
\end{equation}

For this problem, we first generate all primes below the given limit. Then for each prime $p \ge 7$, test whether it satisfies condition \eqref{eq:133.3}. Include in the answer primes that don't satisfy the condition. Also include primes 2, 3, and 5.

\complexity

Let $M = 100000$ be the limit of the primes to test. For each number, primality testing takes $\BigO(\ln M)$; factorizing $(p-1) = 2^r \cdot 5^s$ takes $\BigO(\ln M)$; and testing condition \eqref{eq:133.3} takes $\BigO(\ln M)$.

Time complexity: $\BigO(M \ln M)$.

Space complexity: $\BigO(1)$.

\answer

453647705

\end{document}
