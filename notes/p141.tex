\documentclass[10pt]{article}
\usepackage{amsmath}
\usepackage[utf8]{inputenc}
\setlength{\parskip}{1ex plus 0.5ex minus 0.2ex}

%\newcommand{\problem}[2][No]{\noindent{\large \sc Problem #1. #2.}}
\newcommand{\problem}[2][No]{\noindent{\sc Problem #1. #2.}}
\newcommand{\solution}{\noindent{\sc Solution}}
\newcommand{\method}[1]{\noindent{\sc Method #1}}
\newcommand{\complexity}{\noindent{\sc Complexity}}
\newcommand{\answer}{\noindent{\textsc{Answer}}}
\newcommand{\reference}{\noindent{\sc Reference}}
\newcommand{\seealso}{\noindent{\sc See Also}}

\newcommand{\BigO}{\mathcal{O}}


\begin{document}

\problem[141]{Investigating progressive numbers, $n$, which are also square}

A positive integer, $n$, is divided by $p$ and the quotient and remainder are $q$ and $r$ respectively. In addition $p$, $q$, and $r$ are consecutive positive integer terms in a geometric sequence, but not necessarily in that order.

For example, 58 divided by 6 has quotient 9 and remainder 4. It can also be seen that 4, 6, 9 are consecutive terms in a geometric sequence (common ratio $3/2$). We will call such numbers, $n$, progressive.

Some progressive numbers, such as 9 and $10404 = 102^2$, happen to also be perfect squares.
The sum of all progressive perfect squares below one hundred thousand is 124657.

Find the sum of all progressive perfect squares below one trillion ($10^{12}$).

\solution

Let $n^2 = pq + r$, $p > r > 0$, be a progressive perfect square. We first investigate in what ways can $p, q, r$ form a geometric sequence. There are three cases:

Case 1. $p > q > r$, then we $q^2 = pr$.

Case 2. $q > p > r$, then we can exchange $p$ and $q$ and it becomes case 1.

Case 3. $p > r > q$, then we have $r^2 = pq$. It follows that $n^2 = r^2+r = r(r+1)$. Obviously this can not be a perfect square.

This leaves us with only case one. Rewrite
\[
p = \frac{q^2}{r}
\]
and we note that in order for $p$ to be an integer, $r$ must divide $q^2$. This means $q$ must contain all prime factors of $r$ with sufficient exponent. More precisely, write the factorization of $r$ as
\[
r = r_1^{l_1} \cdots r_k^{l_k} ,
\]
then $q$ must at least contain
\[
q_0 = r_1^{\lceil l_1/2 \rceil} \cdots r_k^{\lceil l_k/2 \rceil} .
\]
Since $q$ must also be greater than $r$, the minimum possible value of $q$ is
\[
q = \left( \left\lfloor \frac{r}{q_0} \right\rfloor + 1\right) q_0 .
\]

We enumerate each $r \ge 1$, and then enumerate each possible $q$ given $r$ until $(pq+r)$ exceeds the limit. For each candidate number, we check whether it is a perfect square. It turns out that the number of candidates grow roughly at a speed of $N^{0.58}$. So the algorithm finishes relatively quickly.

\complexity

Time complexity: $\BigO(N^{0.58})$ (empirical)

Space complexity: $\BigO(1)$

\answer

878454337159

\end{document}
