\documentclass[10pt]{article}
\usepackage{amsmath}
\usepackage[utf8]{inputenc}
\setlength{\parskip}{1ex plus 0.5ex minus 0.2ex}

%\newcommand{\problem}[2][No]{\noindent{\large \sc Problem #1. #2.}}
\newcommand{\problem}[2][No]{\noindent{\sc Problem #1. #2.}}
\newcommand{\solution}{\noindent{\sc Solution}}
\newcommand{\method}[1]{\noindent{\sc Method #1}}
\newcommand{\complexity}{\noindent{\sc Complexity}}
\newcommand{\answer}{\noindent{\textsc{Answer}}}
\newcommand{\reference}{\noindent{\sc Reference}}
\newcommand{\seealso}{\noindent{\sc See Also}}

\newcommand{\BigO}{\mathcal{O}}


\begin{document}

\problem[132]{Prime Factors of a Large Repunit}

A number consisting entirely of ones is called a \emph{repunit}. We shall define $R(n)$ to be a repunit of length $n$.

For example, $R(10) = 1111111111 = 11 \times 41 \times 271 \times 9091$, and the sum of these prime factors is 9414.

Find the sum of the first forty prime factors of $R(10^9)$.

\solution

Given a prime number $p$, and check whether it divides $R(n)$. If $p=3$, then $p$ divides $R(n)$ if and only if the sum of the digits in $R(n)$, $n$, is a multiple of 3. If $p \ne 3$,write $R(n)$ as
\begin{equation}
R(n) = \sum_{k=0}^{n-1} 10^k = \frac{10^n-1}{9} .
\end{equation}
And $p$ divides $R(n)$ if and only if
\begin{equation}
10^n - 1 \equiv 0 \pmod{p} . \label{eq:132.2}
\end{equation}

Let $k$ denote the smallest positive integer such that $10^k \equiv 1 \pmod{p}$. Then equation \eqref{eq:132.2} holds if and only if $n$ is a multiple of $k$. On the other hand, note that 2 and 5 are not prime factors of a repunit. For prime numbers coprime to 10, we have $10^{p-1} \equiv 1 \pmod{p}$ according to Fermat's little theorem. Therefore, $(p-1)$ must also be a multiple of $k$.

Since both $n$ and $(p-1)$ are multiples of $k$, so is $\gcd(n,p-1)$. It follows that
\begin{equation}
10^{\gcd(n,p-1)} \equiv 1 \pmod{p} . \label{eq:132.3}
\end{equation}
It is easy to see that the converse is true: if equation \eqref{eq:132.3} holds, then $p$ is a prime factor of $R(n)$.

Hence, to find the first $m$ prime factors of $R(n)$, we can generate primes $p$ in order and check whether $p$ satisfies \eqref{eq:132.3}. Repeat until we have found $m$ such primes.

\complexity

Let $P$ be the $m$-th prime factor of $R(n)$. It takes $\BigO(P \ln P)$ to generate all the primes below $P$. For each prime, testing equation \eqref{eq:132.3} takes $\BigO(\ln P)$ to compute $\gcd(n,p-1)$ and $\BigO(\ln P)$ to compute modular exponentiation.

Time complexity: $\BigO(P (\ln P)^2)$.

Space complexity: $\BigO(1)$.

\answer

843296

\reference

http://en.wikipedia.org/wiki/Fermat's\_little\_theorem

http://en.wikipedia.org/wiki/Repunit

\end{document}
