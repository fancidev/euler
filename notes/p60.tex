\documentclass[10pt]{article}
\usepackage{amsmath}
\usepackage[utf8]{inputenc}
\setlength{\parskip}{1ex plus 0.5ex minus 0.2ex}

%\newcommand{\problem}[2][No]{\noindent{\large \sc Problem #1. #2.}}
\newcommand{\problem}[2][No]{\noindent{\sc Problem #1. #2.}}
\newcommand{\solution}{\noindent{\sc Solution}}
\newcommand{\method}[1]{\noindent{\sc Method #1}}
\newcommand{\complexity}{\noindent{\sc Complexity}}
\newcommand{\answer}{\noindent{\textsc{Answer}}}
\newcommand{\reference}{\noindent{\sc Reference}}
\newcommand{\seealso}{\noindent{\sc See Also}}

\newcommand{\BigO}{\mathcal{O}}


\newcommand{\vC}{\mathbf{C}}

\begin{document}

\problem[60]{Five Primes Where Any Two Concatenated Is Prime}

The primes 3, 7, 109, and 673, are quite remarkable. By taking any two primes and concatenating them in any order, the result will always be prime. For example, taking 7 and 109, both 7109 and 1097 are prime. The sum of these four primes, 792, represents the lowest sum for a set of four primes with this property.
 
Find the lowest sum for a set of five primes for which any two primes concatenate to produce another prime.

\solution

Suppose that somehow we know that the required set of five primes can be found from a candidate set containing the first $n$ primes, $P = \{ p_1, p_2, \ldots, p_n \}$. We can then formalize the set $P$ as a graph of $n$ vertices, with vertex $i$ corresponding to $p_i$. If concatenating $p_i$ and $p_j$ in both order forms a prime, then we link vertex $i$ and $j$ in the graph. Hence, the problem is transformed into the classical problem of finding a clique (complete subgraph) of five vertices. In addition, we like to minimize the weight of the clique, which is equal to the sum of the primes on its vertices. 

There is no known algorithm that solves this \emph{clique problem} in better than exponential time. The best known method is the Bron-Kerbosch algorithm with time complexity $\BigO(3^{n/3})$. However, for this problem we also need to minimize the weight of the clique, so the situation becomes a bit more complicated. Moreover, there is no good way to know in advance how many primes to include in the candidate set $P$; we may have to resort to trial and error.

Because of these complications, we instead tailor our own algorithm for this problem. Keep a list of all the cliques generated so far. We will call a clique with $k$ vertices a $k$-clique. In particular, the group of 1-cliques consists of the primes themselves, and the group of 2-cliques consists of all the edges in the graph, i.e. pairs of primes that concatenate to another prime. In addition, we denote by $w(c)$ the weight of a clique, $c$.

The outline of the algorithm is as follows. Iterate each prime in order. For each prime, grow the existing cliques. Repeat this process until at least one five-vertex clique is found and any other five-vertex clique will definitely have a larger weight.

The algorithm is detailed below.

Step 0 [Initialization]. 

Let $K = 5$ be the size of the clique to find. 

Let $w_0 = +\infty$ be the upper bound of the weight of the minimum $K$-clique. 

Let $\vC^k = \phi$ for $1 \le k \le K$ denote the group of candidate $k$-cliques that may eventually grow into the minimum $K$-clique. Let $\vC^0 = \{ \phi \}$ contain an empty clique from which cliques of more vertices may be grown.

Discard the prime 2 because it will never form a clique except of itself.

Step 1 [Add vertex]. Compute the next prime, $p$, and add it to the vertex set. Connect $p$ to each prime $q$ generated so far if $(p | q)$ and $(q | p)$ are both prime, where $|$ is the concatenation operator.

Step 2 [Grow cliques]. For each $k = 0$ to $K-1$, do the following:

For each $k$-clique $c$ in $\vC^k$, if $w(c) + (K-k)p \ge w_0$, then remove $c$ from $\vC^k$. This is to ensure that if a $K$-clique of weight $w_0$ has already been found, we don't (need to) grow a $k$-clique of weight $w(c)$ if the smallest possible weight of a $K$-clique grown from $c$ will exceed $w_0$. If $c$ is not removed, append $(c,p)$ to the $(k+1)$-clique group $\vC^{k+1}$ if $p$ is connected to every vertex in $c$.

Step 3 [Loop or terminate]. If new $K$-cliques are generated in Step 2, update the upper bound $w_0 = \min_{c \in \vC^K} w(c)$, and go to Step 1. If all clique groups $\vC^0$ through $\vC^{K-1}$ are empty, print $w_0$, the minimum $K$-clique weight, and terminate.

\complexity

Time complexity: $\BigO(?)$.

Space complexity: $\BigO(?)$.

\answer

26033

\reference

http://en.wikipedia.org/wiki/Clique\_problem

http://en.wikipedia.org/wiki/Bron\%E2\%80\%93Kerbosch\_algorithm

%http://citeseerx.ist.psu.edu/viewdoc/download?doi=10.1.1.48.4074\&rep=rep1\&type=pdf

%http://drops.dagstuhl.de/opus/volltexte/2011/2935/pdf/10441.EppsteinDavid.Paper.2935.pdf

\end{document} 