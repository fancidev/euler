\documentclass[10pt]{article}
\usepackage{amsmath}
\usepackage[utf8]{inputenc}
\setlength{\parskip}{1ex plus 0.5ex minus 0.2ex}

%\newcommand{\problem}[2][No]{\noindent{\large \sc Problem #1. #2.}}
\newcommand{\problem}[2][No]{\noindent{\sc Problem #1. #2.}}
\newcommand{\solution}{\noindent{\sc Solution}}
\newcommand{\method}[1]{\noindent{\sc Method #1}}
\newcommand{\complexity}{\noindent{\sc Complexity}}
\newcommand{\answer}{\noindent{\textsc{Answer}}}
\newcommand{\reference}{\noindent{\sc Reference}}
\newcommand{\seealso}{\noindent{\sc See Also}}

\newcommand{\BigO}{\mathcal{O}}


\begin{document}

\problem[74]{Factorial Chains With Sixty Non-Repeating Terms}

The number 145 is well known for the property that the sum of the factorial of its digits is equal to 145:
\[
1! + 4! + 5! = 1 + 24 + 120 = 145 .
\]

Perhaps less well known is 169, in that it produces the longest chain of numbers that link back to 169; it turns out that there are only three such loops that exist:
\begin{align}
  &169 \rightarrow 363601 \rightarrow 1454 \rightarrow 169 \notag \\
  &871 \rightarrow 45361 \rightarrow 871 \notag \\
  &872 \rightarrow 45362 \rightarrow 872 \notag 
\end{align}

It is not difficult to prove that \emph{every} starting number will eventually get stuck in a loop. For example,
\begin{align}
&69 \rightarrow 363600 \rightarrow 1454 \rightarrow 169 \rightarrow 363601 (\rightarrow 1454) \notag \\
&78 \rightarrow 45360 \rightarrow 871 \rightarrow 45361 (\rightarrow 871)\notag \\
&540 \rightarrow 145 (\rightarrow 145)\notag
\end{align}

Starting with 69 produces a chain of five non-repeating terms, but the longest non-repeating chain with a starting number below $10^6$ contains 60 terms.

How many chains, with a starting number below $10^6$, contain exactly 60 non-repeating terms?
 
\solution

Let $f(n)$ be the sum of factorial of the digits in $n$. If $n$ contains $d$ digits, then the maximum value of $f(n)$ is $9! \times d = 362880 \times d$, and this maximum is achieved when $n = 10^d - 1$.

Hence it is clear that $f(n) \le 362880 \times \lceil \log_{10} n \rceil $. The right-hand side is less than $n$ when $n > 7 \times 9! = 2540160$. Therefore, if we repeatedly apply $f(n)$ to form a chain, the terms will eventually become less than or equal to $7 \times 9!$, and the terms that follow will never go above it. Since there are finite number of integers below $7 \times 9!$, the chain will eventually loop.

For this problem, since we need to compute the length of factorial chain for all starting numbers below $10^6$, we maintain a cache of the chain length of known starting numbers. Since the intermediate terms may go above $10^6$, we allocate the cache to hold $7 \times 9!$ elements. After an unknown starting number has been computed for its factorial chain, the cache for all terms in this chain is updated. A separate list is used to store the chain being computed. 

After we compute all starting numbers below $10^6$, we count the number of chains of length 60 and print the result.

\complexity

Time complexity: $\BigO(N)$.

Space complexity: $\BigO(N)$.

\answer

402

\end{document} 