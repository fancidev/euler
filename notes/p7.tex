\documentclass[10pt]{article}
\usepackage{amsmath}
\usepackage[utf8]{inputenc}
\setlength{\parskip}{1ex plus 0.5ex minus 0.2ex}

%\newcommand{\problem}[2][No]{\noindent{\large \sc Problem #1. #2.}}
\newcommand{\problem}[2][No]{\noindent{\sc Problem #1. #2.}}
\newcommand{\solution}{\noindent{\sc Solution}}
\newcommand{\method}[1]{\noindent{\sc Method #1}}
\newcommand{\complexity}{\noindent{\sc Complexity}}
\newcommand{\answer}{\noindent{\textsc{Answer}}}
\newcommand{\reference}{\noindent{\sc Reference}}
\newcommand{\seealso}{\noindent{\sc See Also}}

\newcommand{\BigO}{\mathcal{O}}


\begin{document}

\problem[7]{Find the 10001st Prime}

By listing the first six prime numbers: 2, 3, 5, 7, 11, and 13, we can see that the 6th prime is 13.

What is the 10001st prime number?

\solution

While the scale of the problem is small and a simple trial division algorithm would produce the answer in no time, we present here a slightly better algorithm that employs the Sieve of Eratosthenes method.

A key result that enables us to implement a sieving method is the following inequality regarding the $n$th prime number, $p_n$:
\[
n \ln n + n \ln \ln n - n < p_n < n \ln n + n \ln \ln n \text{ for $n \ge 6$}.
\]
This inequality provides an upper bound on the numbers that need to be sieved to find the $n$th prime. What follows is just the standard Sieve of Eratosthenes.

\answer

104743

\complexity

Time complexity: $\BigO(n \ln n)$

Space complexity: $\BigO(n \ln n)$

\reference

http://en.wikipedia.org/wiki/Prime\_number\_theorem

\end{document}